%%
%% This is file `sample-sigconf.tex',
%% generated with the docstrip utility.
%%
%% The original source files were:
%%
%% samples.dtx  (with options: `sigconf')
%% 
%% IMPORTANT NOTICE:
%% 
%% For the copyright see the source file.
%% 
%% Any modified versions of this file must be renamed
%% with new filenames distinct from sample-sigconf.tex.
%% 
%% For distribution of the original source see the terms
%% for copying and modification in the file samples.dtx.
%% 
%% This generated file may be distributed as long as the
%% original source files, as listed above, are part of the
%% same distribution. (The sources need not necessarily be
%% in the same archive or directory.)
%%
%%
%% Commands for TeXCount
%TC:macro \cite [option:text,text]
%TC:macro \citep [option:text,text]
%TC:macro \citet [option:text,text]
%TC:envir table 0 1
%TC:envir table* 0 1
%TC:envir tabular [ignore] word
%TC:envir displaymath 0 word
%TC:envir math 0 word
%TC:envir comment 0 0
%%
%%
%% The first command in your LaTeX source must be the \documentclass command.
\documentclass[sigconf]{acmart}
\usepackage{fixme}
%%
%% \BibTeX command to typeset BibTeX logo in the docs
\AtBeginDocument{%
  \providecommand\BibTeX{{%
    \normalfont B\kern-0.5em{\scshape i\kern-0.25em b}\kern-0.8em\TeX}}}

%% Rights management information.  This information is sent to you
%% when you complete the rights form.  These commands have SAMPLE
%% values in them; it is your responsibility as an author to replace
%% the commands and values with those provided to you when you
%% complete the rights form.
%\setcopyright{acmcopyright}
%\copyrightyear{2023}
%\acmYear{2023}


%% These commands are for a PROCEEDINGS abstract or paper.
\acmConference[GECCO '23]{Genetic Evolutionary Computation}{2023}{Lisbon, Portugal}
\acmBooktitle{GECCO '23: xx, xx, xx}
%\acmPrice{15.00}
\acmISBN{xxxxx}

\acmDOI{xxxx}
%%
%% Submission ID.
%% Use this when submitting an article to a sponsored event. You'll
%% receive a unique submission ID from the organizers
%% of the event, and this ID should be used as the parameter to this command.
%%\acmSubmissionID{123-A56-BU3}

%%
%% The majority of ACM publications use numbered citations and
%% references.  The command \citestyle{authoryear} switches to the
%% "author year" style.
%%
%% If you are preparing content for an event
%% sponsored by ACM SIGGRAPH, you must use the "author year" style of
%% citations and references.
%% Uncommenting
%% the next command will enable that style.
%%\citestyle{acmauthoryear}

%%
%% end of the preamble, start of the body of the document source.
\begin{document}

%%
%% The "title" command has an optional parameter,
%% allowing the author to define a "short title" to be used in page headers.
\title{On solving $k$-Domination problem: Variable neighborhood search algorithm and its evolutionary-based control parameters determination}

%%
%% The "author" command and its associated commands are used to define
%% the authors and their affiliations.
%% Of note is the shared affiliation of the first two authors, and the
%% "authornote" and "authornotemark" commands
%% used to denote shared contribution to the research.
\author{Milan Predojević}
\authornote{Both authors contributed equally to this research.}
\email{milan.predojevic@pmf.unibl.org}
\orcid{xxxx-xxxx-xxxx}
\author{M.P.}
%\authornotemark[1]
%\email{webmaster@marysville-ohio.com}
\affiliation{%
  \institution{Faculty of Sciences and Mathematics, University of Banja Luka}
  \streetaddress{Mladen Stojanovi\'c 2}
  \city{Banja Luka}
  \state{Serb Republic}
  \country{Bosnia and Herzegovina}
  \postcode{78000}
}

\author{Aleksandar Kartelj}
\author{A.K.}
\affiliation{%
  \institution{Faculty of Mathematics, Univeristy of Belgrade}
  \streetaddress{--}
  \city{Belgrade}
  \country{Serbia}}
\authornote{Both authors contributed equally to this research.}
\email{ kartelj@math.rs}

\author{Marko Djukanović}
 
\email{marko.djukanovic@pmf.unibl.org}
\orcid{xxxx-xxxx-xxxx}
\author{M.D.}
%\authornotemark[1]
%\email{webmaster@marysville-ohio.com}
\affiliation{%
	\institution{Faculty of Sciences and Mathematics, University of Banja Luka}
	\streetaddress{Mladen Stojanovi\'c 2}
	\city{Banja Luka}
	\state{Serb Republic}
	\country{Bosnia and Herzegovina}
	\postcode{78000}
}
 
 
 
 

%%
%% By default, the full list of authors will be used in the page
%% headers. Often, this list is too long, and will overlap
%% other information printed in the page headers. This command allows
%% the author to define a more concise list
%% of authors' names for this purpose.
\renewcommand{\shortauthors}{Predojevic et al.}

%%
%% The abstract is a short summary of the work to be presented in the
%% article.
\begin{abstract}
  TODO
\end{abstract}

%%
%% The code below is generated by the tool at http://dl.acm.org/ccs.cfm.
%% Please copy and paste the code instead of the example below.
%%



%\ccsdesc[500]{Computer systems organization~Embedded systems}
%\ccsdesc[300]{Computer systems organization~Redundancy}
%\ccsdesc{Computer systems organization~Robotics}
%\ccsdesc[100]{Networks~Network reliability}

%%
%% Keywords. The author(s) should pick words that accurately describe
%% the work being presented. Separate the keywords with commas.
\keywords{variable neighborhood search, graph domination, evolutionary algorithms,  hyperparameters tuning}

%% A "teaser" image appears between the author and affiliation
%% information and the body of the document, and typically spans the
%% page.
%\begin{teaserfigure}
%  \includegraphics[width=\textwidth]{sampleteaser}
%  \caption{Seattle Mariners at Spring Training, 2010.}
%  \Description{Enjoying the baseball game from the third-base
%  seats. Ichiro Suzuki preparing to bat.}
%  \label{fig:teaser}
%\end{teaserfigure}

%%
%% This command processes the author and affiliation and title
%% information and builds the first part of the formatted document.
\maketitle

\section{Introduction}
\fxnote{TODO}
The k-domination problem~\cite{corcoran2021heuristics}...

\section{Problem definition }
    
   
   
   \subsection{The general search space}
   
   
\section{The proposed algorithm}
 
   \subsection{Variable neighborhood search}
   
   \subsection{Fitness function}
  
   \subsection{Shaking}
   
   \subsection{Local search}
 

\section{Experimental evaluation}


\subsection{Parameters tuning}

\subsection{Numerical results }
 
\section{Conclusions and future work}
 TODO
 
\section*{Acknowledgments} 
TODO
%%
%% The next two lines define the bibliography style to be used, and
%% the bibliography file.
\bibliographystyle{ACM-Reference-Format}
\bibliography{gecco_poster_literature}

%%
%% If your work has an appendix, this is the place to put it.
%\appendix
 
\end{document}
\endinput
%%
%% End of file `sample-sigconf.tex'.
